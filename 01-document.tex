% !center 124 | frame 124 -f'\%-\% '
%------------------------------------------------------------------------------------------------------------------------------%
%                                                          Title Page                                                          %
%------------------------------------------------------------------------------------------------------------------------------%
\thispagestyle{empty} % Removes page numbering from the first page
\flushbottom % Makes all text pages the same height
\maketitle % Print the title and abstract box


% !center 124 | frame 124 -f'\%-\% '
%------------------------------------------------------------------------------------------------------------------------------%
%                                                           Abstract                                                           %
%------------------------------------------------------------------------------------------------------------------------------%
\begin{abstract}
    Este estudo inicia propondo \emph{definições} para escatologias (i)~``segundo Deus, em  justiça  e  retidão  procedentes  da
    verdade,'' e as (ii)~``com erro.'' O novo Teorema da ``Dicotomia Escatológica'' é proposto e provado, pelas Escrituras, pelo
    qual conclui-se que as definições contém, em união,  \emph{todas}  as  escatologias,  sendo  as  ``com  erro''  aquelas  que
    \emph{devem ser abandonadas}, e aquela única ``segundo Deus,'' em seus vários escopos, constituir  as  que  \emph{devem  ser
    guardadas, cridas, ensinadas e pro- clamadas} por servos de Jesus Cristo. Estes conceitos introdutórios  são,  no  Apêndice,
    generalizados para \emph{qualquer doutrina bíblica}. Em seguida é proposta a identificação e enunciação de  \emph{princípios
    bíblicos} necessários, porém não suficientes, a escatologias ``segundo Deus.'' Mostra-se que tais princípios,  à  medida  em
    que vão sendo  identificados,  constituem  em  instrumentos  progressivamente  mais  completos  para  \emph{comprovação}  de
    escatologias ``com erro,'' através da \emph{qualificação} e  \emph{exposição},  cabal,  de  erros  incorridos  em  uma  dada
    escatologia sob análise. Além disso, os princípios bíblicos que forem sendo identificados nas  Escrituras  constituem-se  em
    balizadores objetivos do que se deve fazer ou evitar no desenvolvimento de estudos escatológicos. Procede-se,  então,  neste
    estudo, à identificação de 9 (nove) tais princípios e 1 (um) corolário, deduzidos à partir das Escrituras, sob os axiomas de
    que as Escrituras são verdade e Palavra do único Deus verdadeiro. As deduções feitas aqui empregam as  Escrituras  de  forma
    breve e minimalista; assim, defesas e/ou exposições mais abrangentes podem ser feitas separadamente. Implicações  mínimas  e
    imediatas são feitas sobre (a)~a pluralidade posições escatológicas atualmente existente,  (b)~as  linhas  de  interpretação
    alegóricas de profecias, e (c)~as linhas e  argumentos  de  interpretação  que  empregam  passagens  futuras  para  explicar
    termos-chave de passagens anteriores. Para a glória de  Deus,  espera-se  que  a  abordagem  e  os  conceitos  deste  estudo
    contribuam para a unificação da escatologia bíblica \bQuote{segundo Deus, em  justiça  e  retidão  procedentes  da  verdade}
    Ef~4.24 (ARA).
\end{abstract}

\ENtxt{
    \begin{abstract}
        Here goes the abstract.
    \end{abstract}
}


% !center 124 | frame 124 -f'\%-\% '
%------------------------------------------------------------------------------------------------------------------------------%
%                                                           License                                                            %
%------------------------------------------------------------------------------------------------------------------------------%
\section*{Licença}

    \scriptsize\noindent%
    \begin{minipage}{\columnwidth}
        \centering\tt
        \includegraphics[height=6.0mm]{cc/by-nc.pdf}\\[0.5\smallskipamount]
        {\scriptsize\texttt{https://creativecommons.org/licenses/by-nc/4.0/}}
    \end{minipage}
    \normalsize


% !center 124 | frame 124 -f'\%-\% '
%------------------------------------------------------------------------------------------------------------------------------%
%                                                      Table Of Contents                                                       %
%------------------------------------------------------------------------------------------------------------------------------%
\tableofcontents


% !center 124 | frame 124 -f'\%-\% '
%------------------------------------------------------------------------------------------------------------------------------%
%                                                     Measure Adjustments                                                      %
%------------------------------------------------------------------------------------------------------------------------------%
\setlength{\parskip}{0.5\baselineskip}


% !center 124 | frame 124 -f'\%-\% '
%------------------------------------------------------------------------------------------------------------------------------%
%                                                         Introduction                                                         %
%------------------------------------------------------------------------------------------------------------------------------%

\section{Introdução}

    O assunto de \emph{escatologia bíblica} --- que é o estudo  das  profecias  bíblicas,  ou,  etimologicamente,  a  junção  de
    \GRtxt{ἔσχατος},  que,  segundo  Bailly~\cite[pp.~817--8]{2000-BaillyA-Hachette},  significa:  ``\textit{o   que   está   na
    extremidade,  no  extremo,   no   final;   tanto   nos   sentidos   espacial   quanto   temporal;   portanto:   as   últimas
    coisas\/}''\footnote{\textsc{Lit}: \FRtxt{«qui est à l'extrémité, extrème, dernier : \textbf{I}  (avec  idée  de  lieu)  ...
    \textbf{II}   (avec   idée   de   temps)   :    ...    le    dernier    ...    »}.},    juntamente    com    \GRtxt{-λογία}:
    palavras~\cite{1997-ManiatoglouMPF-Porto} (estudo), a saber: o estudo das últimas coisas, com base na Bíblia --- mostra,  na
    atualidade, uma grande variedade de visões de mundo, com vertiginosas disparidades e irreconciliáveis incompatibilidades  de
    conclusões a que chegam os diferentes estudos, os quais, \emph{supostamente}, empregaram as mesmas Escrituras como base.

    Percebe-se um contraste aberrante entre  aquilo  que  as  Escrituras  Sagradas  revelam  acerca  do  \bLineQuote{único  Deus
    verdadeiro}{Jo~17.3}{ARA} com o atual estado de coisas da escatologia bíblica. Di\-a\-gnos\-ti\-ca-se, com isso, não  apenas
    um cenário lamentável para a cristandade, mas também um atestado dos efeitos de  uma  batalha  entre  luz  e  trevas,  entre
    verdade e engano, no qual o engano parece estar colhendo do muito que semeou.

    Porém, as Escrituras trazem a seguinte exortação:

        \bBlockQuote{% !j -i12 116
            esforçando-vos diligentemente por preservar a \textbf{unidade do Espírito} no vínculo da paz; há somente
            \textbf{um corpo} e \textbf{um Espírito}, como também fostes  chamados  n\textbf{uma  só  esperança}  da
            vossa vocação; há \textbf{um só Senhor}, \textbf{uma só fé}, \textbf{um só batismo}; \textbf{um só  Deus
            e Pai de todos}, o qual é sobre todos, age por meio de todos e \textbf{está em todos}.}{Ef~4.3}{ARA}

    Desta porção, já se pode concluir que a mera existência  de  uma  grande  multiplicidade  de  escatologias,  com  conclusões
    mutuamente incompatíveis e irreconciliáveis \emph{não é fruto da ação de Deus no corpo}, porém,  certamente  do  inimigo  de
    nossas almas, através de homens de artimanha, os quais com astúcia induzem ao erro, conforme o que está escrito:

        \bBlockQuote{% !j -i12 116
            E a \textbf{graça} foi concedida \textbf{a cada um de nós segundo a  proporção  do  dom  de  Cristo}.  E
            \textbf{ele mesmo} concedeu uns para \textbf{apóstolos},  outros  para  \textbf{profetas},  outros  para
            \textbf{evangelistas}  e   outros   para   \textbf{pastores}   e   \textbf{mestres},   com   vistas   ao
            \textbf{aperfeiçoamento dos santos} para o \textbf{desempenho do seu serviço}, para a \textbf{edificação
            do corpo de Cristo}, até que todos cheguemos à \textbf{unidade da fé} e do \textbf{pleno conhecimento do
            Filho de Deus}, à perfeita varonilidade,  à  medida  da  estatura  da  plenitude  de  Cristo,  para  que
            \textbf{não mais} sejamos como \textbf{meninos, agitados de um lado para outro e levados  ao  redor  por
            todo  vento  de  doutrina,  pela   artimanha   dos   homens,   pela   astúcia   com   que   induzem   ao
            erro}.}{Ef~4.7,11--14}{ARA}

    Creio, assim, não apenas na \emph{existência}, mas também na \emph{possibilidade} de abordagem da escatologia (assim como de
    qualquer assunto da Palavra de Deus), pelo dom de Cristo, de modo a formar o são e correto ensino; a chegar  nas  conclusões
    verdadeiras, no sentido verdadeiro das profecias, \bLineQuote{segundo o propósito daquele que faz todas as coisas conforme o
    conselho da sua vontade}{Ef~1.11}{ARA}, pois que também a Escritura testifica, por meio de Moisés:

        \bBlockQuote{% !j -i12 116
            Porque este mandamento que, hoje, te ordeno não é demasiado difícil,  \textbf{nem  está  longe}  de  ti.
            \textbf{Não está nos céus}, para dizeres: Quem subirá por nós aos céus, que no-lo  traga  e  no-lo  faça
            ouvir, para que o cumpramos? \textbf{Nem está além do mar}, para dizeres: Quem passará por nós  além  do
            mar que no-lo traga e no-lo faça ouvir, para que o cumpramos? Pois \textbf{esta palavra está  mui  perto
            de ti}, na tua boca e no teu coração, para a cumprires.}{Dt~30.11-14}{ARA}


    %---------------------------------------------------------------------------------------------------------------------------
    \subsection{Escatologia ``Segundo Deus'' --- Definições}

    Tem-se em mente uma abordagem escatológica  \bQuote{segundo\linebreak{}  Deus}\footnote{A  expressão  \bQuote{segundo  Deus}
    também aparece em 1Pe~4.6 e três vezes em 2Co~7.9-11.}, feita ``à maneira de  Deus;''  aquela  que,  baseada  unicamente  em
    verdade, é conduzida em retidão e chega à verdade, a saber, ao que o  próprio  Deus  tem  reservado  para  o  futuro  e  vem
    anunciando desde o princípio:

    % !j 116 -i8
    \begin{DEF}[Escatologia ``Segundo Deus'']
        \label{def.esc.segDeus}
        A escatologia  \bQuote{segundo  Deus}  é  aquela  feita  \bLineQuote{segundo  Deus,  em  justiça  e  retidão
        procedentes da verdade}{Ef~2.24}{ARA}.
    \end{DEF}

    Passa-se também à definição de escatologia com \emph{erro} --- entendido como qualquer violação direta de qualquer um ou mais
    preceitos das Escrituras --- ou escatologia com engano, ou enganosa, que não é segundo Deus:

    % !j 116 -i8
    \begin{DEF}[Escatologia Com Erro]
        \label{def.esc.comerro}
        Seja a escatologia com erro, aquela que incorre em erro, seja nas suas premissas, ou nos  seus  métodos,  ou
        nos seus processos, ou nas suas conclusões.
    \end{DEF}

    De acordo com Daepp e Gorkin, podemos enunciar uma sentença que \emph{sempre é verdadeira}, por meio de um
    teorema~\cite[p.~17]{2011-DaeppU+GorkinP-Springer}:

    % !j 116 -i8
    \begin{THE}[Dicotomia Escatológica]
        \label{the.esc.dicotomia}
        Qualquer escatologia será ou \bQuote{segundo Deus}, ou ``com erro.''
    \end{THE}

    % !j 116 -i8
    \begin{proof}[Prova do Teorema da Dicotomia Escatológica]
        \label{pro.the.dicotomia}

        Seja uma escatologia $\epsilon_i$ qualquer, com $i \in \{1, 2, 3, \ldots\}$ sendo um  índice  que  numera-a,
        identificando-a de forma única e inequívoca.

        Caso (i)~$\epsilon_i$ incorra em erro  (contenha  erro),  será,  pela  Definição~\ref{def.esc.comerro},  uma
        escatologia ``com erro.'' Neste caso, tal escatologia \emph{não poderá} ser \bQuote{segundo Deus}, pois está
        escrito:

        \bBlockQuote{% !j -i12 116
            Não vos escrevi porque não saibais a verdade; antes, porque a sabeis, e  porque  \textbf{mentira  alguma
            jamais procede da verdade}.}{1Jo~2.21}{ARA}

        \noindent violando a Definição~\ref{def.esc.segDeus}, pelo erro jamais proceder da verdade.

        Caso (ii)~$\epsilon_i$ não contiver erro, não incorrer em erro, terá sido feita \bLineQuote{segundo Deus, em
        justiça    e    retidão    procedentes    da    verdade}{Ef~2.24}{ARA}    e     será,     portanto,     pela
        Definição~\ref{def.esc.segDeus}, escatologia \bQuote{segundo Deus}.

        Neste caso, tal escatologia não poderá ser ``com erro,'' pois a premissa de não conter erro viola a
        Definição~\ref{def.esc.comerro}.

    \end{proof}

    Uma vez provado o Teorema~\ref{the.esc.dicotomia}, seguem-se os dois Corolários\footnote{Corolários são consequências
    lógicas de um Teorema.} abaixo:

    % !j 116 -i8
    \begin{COR}
        \label{cor.dicotomia.1}
        Nenhuma escatologia pode ser simultaneamente \bQuote{segundo Deus} e ``com erro.''
    \end{COR}

    % !j 116 -i8
    \begin{COR}
        \label{cor.dicotomia.2}
        A união das Definições~\ref{def.esc.segDeus} e~\ref{def.esc.comerro} contém todas as escatologias.
    \end{COR}

    As Definições~\ref{def.esc.segDeus} e~\ref{def.esc.comerro} dadas são úteis  na  \emph{classificação}  de  escatologias  que
    \textbf{devem  ser  abandonadas}  por  membros  do  corpo   de   Cristo   (à   luz   de   Ef~4.7,11--14,   já   citado),   e
    aquelas\footnote{Adianto aqui que a pluralidade das escatologias \bQuote{segundo Deus} é  devida  unicamente  em  função  do
    escopo, e não do teor, meramente para fazer o argumento.}, \bQuote{segundo Deus}, que \textbf{deve ser  guardada}  e  também
    \emph{crida, ensinada e proclamada}.

    Importa notar que tudo o que foi estabelecido até este ponto pode ser \textbf{\textsc{generalizado  para  qualquer  doutrina
    bíblica}}, desde as Definições, incluindo o Teorema, sua Prova e seus dois Corolários.  Esta  generalização  é  colocada  no
    Apêndice.

    Muito embora sejam úteis as Definições~\ref{def.esc.segDeus} e~\ref{def.esc.comerro} dadas, elas não  são,  necessariamente,
    de fácil ou consensual aplicação, especialmente em um cenário --- o atual --- no  qual  proliferam,  não  apenas  diferentes
    visões de mundo, mas igualmente, erros grosseiros nas diferentes  abordagens  escatológicas,  além  de  evidência  de  muito
    desapreço por exatidão e verdade.

    De  um  ponto  de  vista  prático,  se  um  número  razoável  de  \emph{princípios}  for  identificado,   tais   que   sejam
    (a)~\emph{indispensáveis} a estudos proféticos \bQuote{segundo Deus}, e que também sejam (b)~de fácil demonstração quanto  à
    sua violação; ter-se-á \textsc{instrumentos para comprovação de escatologias ``com erro''}, uma vez que o caso~(i) da  Prova
    do Teorema da Dicotomia Escatológica, classifica inequivocamente a escatologia sob análise como sendo ``com erro.''

    Assim, caso uma escatologia $\epsilon_i$ viole \emph{qualquer um} dos princípios reunidos,  os  quais  são,  por  definição,
    \emph{indispensáveis} a estudos proféticos \bQuote{segundo  Deus},  qualifica-se  e  expõe-se,  \textbf{cabalmente}  o  erro
    incorrido pela escatologia $\epsilon_i$ em questão!

    As vantagens da presente abordagem incluem: (1)~\emph{não} se  faz  necessário  deduzir  um  conjunto  \emph{suficiente}  de
    princípios, apenas alguns \emph{necessários} a  estudos  \bQuote{segundo  Deus},  e  (2)~a  busca  \emph{não  necessita  ser
    exaustiva}, podendo ser focada em uma dada escatologia específica sob análise, e podendo  ser  complementada  em  estudos  e
    ocasiões posteriores, e, finalmente, (3)~os princípios reunidos servem de balizadores objetivos do  que  se  deve  fazer  ou
    evitar em estudos escatológicos e/ou proféticos em curso, uma vez que, antes de empreender qualquer abordagem no assunto  de
    profecias, é de extrema importância identificar e pautar-se no que as Escrituras afirmam sobre Deus e sobre  si  mesmas,  em
    conexão ao estudo de profecias.


    %---------------------------------------------------------------------------------------------------------------------------
    \subsection{Objetivos Gerais}

    Assim, este estudo objetiva \emph{identificar} e \emph{enunciar}, de  acordo  com  as  Escrituras,  alguns  \emph{princípios
    bíblicos} que sejam, por definição, \emph{indispensáveis} a estudos proféticos e a escatologia  \bQuote{segundo  Deus},  com
    vistas à exposição de erros em escatologias, podendo classificá-las cabal e inequivocamente como ``escatologias com  erro,''
    de acordo com Definição~\ref{def.esc.comerro} dada, fornecendo, assim, os aludidos \textsc{instrumentos para comprovação  de
    escatologias ``com erro''}.



% !center 124 | frame 124 -f'\%-\% '
%------------------------------------------------------------------------------------------------------------------------------%
%                                    Princípios Bíblicos Para Escatologia ``Segundo Deus''                                     %
%------------------------------------------------------------------------------------------------------------------------------%

\section{Princípios Bíblicos Para Escatologia ``Segundo Deus''}

    %---------------------------------------------------------------------------------------------------------------------------
    \subsection{Axiomas}

    O assunto já delineado será estudado com base nos seguintes axiomas:

    \begin{enumerate}

        \item\label{ax:Deus.exis} Há um só Deus;

        \item\label{ax:Escr.pala} As Escrituras Bíblicas são Palavra deste Deus;

        \item\label{ax:Deus.verd} As Escrituras Bíblicas são verdade.

    \end{enumerate}

    Entende-se por ``Escrituras Bíblicas'' o conjunto coeso de 66 livros, composto pelos 39 livros da Bíblia Hebraica e pelos 27
    livros do Novo Testamento Cristão.


    %---------------------------------------------------------------------------------------------------------------------------
    \subsection{Da Unicidade da Realidade do Princípio ao Fim}

    Observa-se que as Escrituras \emph{sempre são assertivas} em relação (i)~à  \emph{realidade}  e  (ii)~à  \emph{história},  a
    exemplo de:

        \bBlockQuote{% !j -i12 116
            E fez Deus a expansão e fez separação entre as águas que estavam debaixo da  expansão  e  as  águas  que
            estavam sobre a expansão. \textbf{E assim foi}.}{Gn~1.7}{ARC}

    A sentença: \bQuote{E assim foi,} indica uma \textbf{realidade e história únicas}  ---  ``assim,''  e  não  de  outra  forma
    diferente ou paralela. Além disso, pelas Escrituras, Deus afirma, \emph{assertivamente}, quanto (iii)~ao \emph{futuro},  por
    meio de Isaías:

        \bBlockQuote{% !j -i12 116
            Lembrai-vos das coisas passadas desde a antiguidade: que \textbf{eu sou Deus,  e  não  há  outro  Deus},
            \textbf{não há outro semelhante a mim}; que  \textbf{anuncio  o  fim  desde  o  princípio}  e,  desde  a
            antiguidade, as coisas que ainda não  sucederam;  que  digo:  \textbf{o  meu  conselho  será  firme},  e
            \textbf{farei toda a minha vontade};}{Is~46.9,10}{ARC}

    A capacidade de anunciar, \textbf{acertadamente} \bQuote{coisas que ainda não sucederam} é um  \emph{atributo  exclusivo  de
    Deus, que o distingue de todos os demais seres}, conforme \bQuote{não há outro semelhante a mim}. Ainda, o que Deus anuncia,
    pela sua Palavra, é \bQuote{o fim desde o princípio} ---  note:  ``o  fim,''  no  singular,  e  não  uma  multiplicidade  de
    `possíveis' fins, indicando, assim, um \textbf{futuro único}.

    As Escrituras falam, portanto, de \emph{história única}, de \emph{realidade única} e de \emph{futuro único} --- de modo  que
    o espaço-tempo dos \bQuote{céus e terra} possui \textbf{unicidade}, sem `realidades paralelas'.

    Está demonstrado, então, pelas Escrituras, a \emph{unicidade da realidade do princípio ao fim}, de onde se extrai:

    % !j 116 -i8
    \begin{PRI}[da Unicidade da Realidade do Princípio ao Fim]
        \label{pri.unicidade}
        Existe, de acordo com as Escrituras, apenas uma \textsc{única realidade}, uma \textsc{única história}  e  um
        \textsc{único futuro}, que se realizará.
    \end{PRI}


    %---------------------------------------------------------------------------------------------------------------------------
    \subsection{Da Veracidade Das Profecias Divinas}

    O Senhor Deus, ao reiterar seus atributos a Judá, por meio do profeta Isaías, o faz de forma \emph{taxativa}:

        \bBlockQuote{% !j -i12 116
            Porque assim diz o Senhor, que \textbf{criou os céus}, o Deus que \textbf{formou a terra}, que \textbf{a
            fez e a estabeleceu}; que não a criou para ser um caos, mas para ser habitada: \textbf{Eu sou o  Senhor,
            e não há outro}.}{Is~45.18}{ARA}

    Sabemos, pela Carta aos Romanos, que \bLineQuote{os atributos invisíveis de Deus, assim o seu  \textbf{eterno  poder},  como
    também a sua \textbf{própria divindade}, claramente se reconhecem, desde o princípio do mundo, sendo \textbf{percebidos} por
    meio das \textbf{coisas que foram criadas}.}{Rm~1.20}{ARA}, de sorte que ao declarar-se Autor de céus e terra, quando  falou
    por meio do profeta Isaías, \bQuote{claramente se reconhece} que Deus está a evocar Seus atributos de \bQuote{eterno  poder,
    como também a sua própria divindade}.

    Este Deus de eterno poder continua, por meio do profeta, dizendo:

        \bBlockQuote{% !j -i12 116
            Não falei em \textbf{segredo}, nem em lugar algum de \textbf{trevas} da terra; não disse à  descendência
            de  Jacó:  Buscai-me  em  vão;  \textbf{eu,  o  Senhor,  falo   a   verdade   e   proclamo   o   que   é
            direito}.}{Is~45.19}{ARA}

    Aqui é acrescentado que a revelação de Deus não foi secreta e com o bendito testemunho: \bQuote{eu, o Senhor, \textbf{falo a
    verdade e proclamo o que é direito}}.

    Assim, está diretamente declarado nas Escrituras que as proclamações de Deus por intermédio de seus profetas  ---  a  saber,
    \emph{todas as profecias} --- \emph{são verdade e direito}.

    % !j 116 -i8
    \begin{PRI}[da Veracidade das Profecias Divinas]
        \label{pri.veracidade}
        De acordo com as Escrituras, todas as profecias divinas são verdade e direito.
    \end{PRI}


    %---------------------------------------------------------------------------------------------------------------------------
    \subsection{Da Equivalência Entre Profecias e Promessas Divinas}

        \bBlockQuote{% !j -i12 116
            E assim, depois de esperar com paciência, obteve Abraão a promessa.}{Hb~6.15}{ARA}
        
    A veracidade das profecias divinas implica em certeza de seu cumprimento, portanto \emph{as profecias divinas são  promessas
    divinas}, mas quais é justiça esperar --- \bLineQuote{É o caso de Abraão, que creu em Deus, e isso  lhe  foi  imputado  para
    justiça.}{Gl~3.6}{ARA}.

    Ainda, cita-se:

        \bBlockQuote{% !j -i12 116
            E \textbf{a si mesmo se purifica} todo o  que  nele  tem  esta  \textbf{esperança},  assim  como  ele  é
            puro.}{1Jo~3.3}{ARA}

    E também:

        \bBlockQuote{% !j -i12 116
            \textbf{Bem-aventurados}  aqueles  que  \textbf{leem}  e  aqueles  que  \textbf{ouvem}  as  palavras  da
            \textbf{profecia} e \textbf{guardam} as coisas nela escritas, pois o tempo está próximo.}{Ap~1.3}{ARA}

    As Escrituras declaram e extrai-se o princípio da equivalência entre profecias divinas e promessas divinas:

    % !j 116 -i8
    \begin{PRI}[da Equivalência Entre Profecias e Promessas Divinas]
        \label{pri.promessa}
        De acordo com as Escrituras, todas as profecias divinas são promessas divinas, nas quais é justiça esperar.
    \end{PRI}


    %---------------------------------------------------------------------------------------------------------------------------
    \subsection{Da Verificabilidade Das Profecias Divinas}

    Sendo a realidade única e as profecias promessas sempre verdadeiras; com a passagem do tempo, aquilo que antes era futuro, a
    saber, as \bQuote{coisas que ainda não sucederam}{Is~46.10}{ARC}, uma vez chegado seu tempo e  cumpridas,  podem  ser  assim
    testemunhadas, ou verificadas, pelos homens. Tais exercícios de constatação são frequentemente registrados nas Escrituras:

        \bBlockQuote{% !j -i12 116
            \textbf{Nenhuma promessa falhou} de todas as \textbf{boas palavras  que  o  Senhor  falara}  à  casa  de
            Israel; \textbf{tudo se cumpriu}.}{Js~21.45}{ARA}
 
    \bQuote{Nenhuma promessa falhou} / \bQuote{tudo se cumpriu.} --- as profecias divinas são  verificáveis  a  seu  tempo.  Que
    maravilha!

    Ainda, os Evangelhos atestam muitos cumprimentos de profecias:

        \bBlockQuote{% !j -i12 116
            Ela dará à luz um filho e lhe porás o nome de Jesus, porque ele salvará o seu povo  dos  pecados  deles.
            Ora, tudo isto \textbf{aconteceu para que se cumprisse o que fora dito pelo  Senhor  por  intermédio  do
            profeta}: Eis que a virgem conceberá e dará à luz um filho, e ele será chamado pelo nome de Emanuel (que
            quer dizer: Deus conosco).}{Mt~1.21--23}{ARA}

    E depois:

        \bBlockQuote{% !j -i12 116
            Dispondo-se ele, tomou de noite o menino e sua mãe e partiu para o Egito; e lá  ficou  até  à  morte  de
            Herodes, \textbf{para que se cumprisse o que fora dito pelo  Senhor,  por  intermédio  do  profeta}:  Do
            E\-gi\-to chamei o meu Filho.}{Mt~2.14,15}{ARA}

    E em seguida:

        \bBlockQuote{% !j -i12 116
            Então, \textbf{se cumpriu o que fora dito por intermédio do profeta Jeremias}:  Ouviu-se  um  clamor  em
            Ramá, pranto, [choro] e grande lamento; era Raquel chorando por seus filhos e  inconsolável  porque  não
            mais existem.}{Mt~2.17,18}{ARA}

    E em seguida:

        \bBlockQuote{% !j -i12 116
            E foi habitar numa cidade chamada Nazaré, \textbf{para que se cumprisse o que fora dito  por  intermédio
            dos profetas}: Ele será chamado Nazareno.}{Mt~2.23}{ARA}

    A cada \emph{verificação} do \emph{cumprimento fiel} das \emph{profecias divinas}, \textbf{Deus é  glorificado  como  Deus},
    como o único \bLineQuote{que desde o princípio anuncio o que há de acontecer e desde a antiguidade, as coisas que ainda  não
    sucederam;}{Is~46.10}{ARA}.

    Assim, o princípio da verificabilidade das profecias divinas é abundantemente suportado pelas Escrituras, de onde, então, se
    enuncia:

    % !j 116 -i8
    \begin{PRI}[da Verificabilidade Das Profecias Divinas]
        \label{pri.verificabilidade}
        De acordo com as Escrituras, todas as profecias divinas cum\-prem-se a seu tempo, após  o  qual  é  possível
        verificá-las, conferindo-as com a realidade dos acontecimentos, \bQuote{para que  se  ouça  e  se  diga:
        Verdade é!}
    \end{PRI}

    (Esta última citação é de Is~43.9 (ARA)~\cite{ARA}).


    %---------------------------------------------------------------------------------------------------------------------------
    \subsection{Da Vigilância Divina à Sua Palavra}

    As Escrituras frequentemente explicam que certas coisas vieram a acontecer  com  o  propósito  específico  de  \emph{cumprir
    profecia}, de \emph{cumprir o que está escrito}:

        \bBlockQuote{% !j -i12 116
            Tudo isto, porém, \textbf{aconteceu para que se  cumprissem  as  Escrituras  dos  profetas}.  Então,  os
            discípulos todos, deixando-o, fugiram.}{Mt~26.56}{ARA}

    Eminentemente, temos a visão da vara de amendoeira, dada a Jeremias: \bLineQuote{Veio ainda a palavra do Senhor, dizendo:
    Que vês tu, Jeremias? Respondi: vejo uma vara de amendoeira.}{Jr~1.11}{ARA}, e a resposta divina foi:

        \bBlockQuote{% !j -i12 116
            Disse-me  o  Senhor:  Viste   bem,   porque   \textbf{eu   velo   sobre   a   minha   palavra   para   a
            cumprir.}}{Jr~1.12}{ARA}

    Note-se que `velar' significa: ``permanecer de vigia, de sentinela''~\cite{2009-Houaiss+Franco-Objetiva}. Assim, o Deus  que
    está \bLineQuote{sustentando todas as coisas pela palavra do seu  poder}{Hb~1.3}{ARA},  que  \bQuote{é  antes  de  todas  as
    coisas} e no qual \bLineQuote{tudo subsiste}{Cl~1.17}{ARA}, permanece de sentinela para \textbf{cumprir} Sua Palavra!

    % !j 116 -i8
    \begin{PRI}[da Vigilância Divina à Sua Palavra]
        \label{pri.vigilância}
        De acordo com as Escrituras, Deus mantém constante vigilância à toda a  sua  Palavra,  com  o  propósito  de
        cumprí-la.
    \end{PRI}


    %---------------------------------------------------------------------------------------------------------------------------
    \subsection{Cumprimento Literal ou Alegórico?}

    Para que não haja qualquer dúvida sobre a firmeza do propósito Divino no cumprimento fiel de  suas  promessas  e  profecias,
    tem-se, no Livro de Deuteronômio --- portanto na Lei, a profecia da vinda do  profeta  em  cuja  boca  Deus  colocaria  Suas
    Palavras:

        \bBlockQuote{% !j -i12 116
            Suscitar-lhes-ei um profeta do meio de seus irmãos, semelhante a  ti,  \textbf{em  cuja  boca  porei  as
            minhas palavras}, e ele lhes \textbf{falará tudo o que eu lhe ordenar}.}{Dt~18.18}{ARA}

    A profecia é solene; Deus continua:

        \bBlockQuote{% !j -i12 116
            De todo aquele que não ouvir as \textbf{minhas palavras}, que \textbf{ele falar em meu nome}, disso  lhe
            pedirei contas. Porém o profeta que \textbf{presumir de falar alguma palavra em meu nome, que eu lhe não
            mandei  falar},  ou  o   que   falar   em   nome   de   outros   deuses,   esse   profeta   \textbf{será
            morto}.}{Dt~18.19,20}{ARA}

    Aqui as implicações são seríssimas --- vida ou morte!  ---  Tal  que  se  torna  \emph{absolutamente  imperioso}  distinguir
    adequadamente a Palavra do Senhor daquela de falsos profetas.

    O texto segue, providencialmente, nesta exata direção: \bLineQuote{Se disseres no teu  coração:  \textbf{Como  conhecerei  a
    palavra que o Senhor não falou?}}{Dt~18.21}{ARA}, e a resposta divina \emph{não deixa dúvidas}:

        \bBlockQuote{% !j -i12 116
            \textbf{Sabe que}, quando esse profeta falar em nome do Senhor, e a palavra dele se \textbf{não cumprir,
            nem suceder, como profetizou}, esta é palavra que o Senhor \textbf{não disse};  \textbf{com  soberba,  a
            falou o tal profeta; não tenhas temor dele}.}{Dt~18.22}{ARA}

    O termo \bQuote{Sabe que} é para dar \emph{certeza plena no assunto!} O v.~22 revela qual é a palavra que o Senhor \bQuote{não
    disse} --- a saber, aquela que \bQuote{não cumprir, nem suceder, como profetizou}.

    Aqui Deus propõe um \emph{teste de  realidade},  e  falsas  profecias  não  passam  no  teste  de  realidade,  a  saber:  do
    \emph{cumprimento \textbf{como profetizado}}.

    Nosso interesse é extrair, por meio das Escrituras, conhecimento sobre as profecias de Deus. Se uma profecia é de Deus,  tal
    não pode ser \bQuote{a palavra que o Senhor não falou} do v.~21; por outro lado, \bQuote{a palavra que o Senhor não  falou},
    do v.~21, não pode ser de Deus. Claramente tem-se aqui apenas duas possibilidades: ou a  profecia  (i)~é  de  Deus,  ou  ela
    (ii)~não é. O texto sagrado aqui é \emph{suficiente} para a determinação de todos os dois possíveis casos, pelo  emprego  da
    lógica  mais  elementar  no  entendimento  do  texto.  Se  uma  possibilidade  foi  enunciada,  sua   \emph{negação}   leva,
    necessariamente, à outra.

    Desta forma, tem-se, \emph{com certeza} --- por ser verdade e pelo: \bQuote{Sabe que} do início do v.~22 ---  que  \textbf{a
    palavra que o Senhor disse cumpre-se \textsc{como profetizada}}, de acordo com Dt~18.22! E assim elimina-se, efetivamente  e
    pela Palavra de Deus, qualquer possibilidade de cumprimento aproximado,  genérico,  em  alegoria,  de  sentido,  leitura  ou
    conclusão \emph{diferente de como está escrito}, diferente de \textbf{como foi profetizado}.

    Importa pontuar que a própria profecia do verso~18 cumpriu-se \textbf{\textsc{literalmente}} em Jesus Cristo:
 
        \bBlockQuote{% !j -i12 116
            Não crês que eu estou no Pai e que o Pai está em mim? \textbf{As palavras que eu vos digo  não  as  digo
            por mim mesmo; mas o Pai, que permanece em mim, faz as suas obras}.}{Jo~14.10}{ARA}

    Foi profetizado: \bQuote{em cuja boca porei as minhas palavras}, e cumpriu-se \bQuote{As palavras que eu  vos  digo  não  as
    digo por mim mesmo}, ou seja, \textbf{cumpriu-se como profetizado}!

    E ainda, com relação ao que foi profetizado: \bQuote{ele lhes falará tudo o  que  eu  lhe  ordenar},  temos  o  registro  do
    cumprimento, em Jesus Cristo, assim:

        \bBlockQuote{% !j -i12 116
            Então, lhes falou Jesus: Em verdade, em verdade vos digo que \textbf{o  Filho  nada  pode  fazer  de  si
            mesmo}, senão \textbf{somente} aquilo que vir fazer o Pai; porque \textbf{tudo o que este fizer, o Filho
            também semelhantemente o faz}.}{Jo~5.19}{ARA}

    E ainda:

        \bBlockQuote{% !j -i12 116
            E aquele que me enviou está comigo,  não  me  deixou  só,  porque  \textbf{eu  faço  sempre  o  que  lhe
            agrada}.}{Jo~8.29}{ARA}

    \noindent portanto, novamente \textbf{cumprimento como profetizado}!

    Assim, pelas Escrituras, derivou-se o princípio de que \textbf{profecia de Deus cumpre-se como foi profetizada}; portanto:

    % !j 116 -i8
    \begin{PRI}[do Cumprimento Como Profetizado]
        \label{pri.comoprof}
        De acordo com as Escrituras, todas as profecias divinas cumprem-se como profetizadas.
    \end{PRI}

    Somente Deus sabe \emph{acertadamente} o futuro, e Sua Palavra diz que \textbf{Ele não profetiza coisas que não se cumprirão
    como profetizado} --- afinal, tais são exemplos de \bQuote{palavra que o Senhor \textbf{não disse}}; assim, temos, como
    Corolário do Princípio~\ref{pri.comoprof}:

    % !j 116 -i8
    \begin{COR}[da Alegação de Cumprimento Alegórico]
        \label{def.cor.alegorico}
        De acordo com as Escrituras, não se pode alegar, para qualquer  profecia  divina,  cumprimento  com  sentido
        diferente, diverso de como elas en\-con\-tram-se profetizadas.
    \end{COR}

    Por outro lado, falsas  profecias  é  que  devem  ter  seu  ``entendimento''  elastificado,  generalizado,  maleabilizado  e
    alegorizado para poderem encaixar-se em  \emph{qualquer  coisa}  que  venha  a  acontecer  e  assim,  alegar  cumprimento  e
    legitimidade --- uma manipulação maliciosa necessária a criaturas incapazes de prever acertadamente o  futuro  ---  e  fazer
    isso com profecias divinas, em violação do Corolário~\ref{def.cor.alegorico}, é  blasfêmia  contra  Deus,  ao  reduzí-Lo  ao
    patamar de falso profeta.


    %---------------------------------------------------------------------------------------------------------------------------
    \subsection{Da Genuína Intenção de Deus}

    Lê-se nas Escrituras, que o enviado Paulo falava, da parte do próprio Deus com \emph{sinceridade}:

        \bBlockQuote{% !j -i12 116
            Porque nós não estamos, como tantos outros, mercadejando a palavra de Deus; antes, em Cristo é que falamos na
            presença de Deus, com \textbf{sinceridade} e \textbf{da parte do próprio Deus}.}{2Co~2.17}{ARA}

    Também temos o testemunho de que \textbf{Deus é fiel} e por isso, Sua palavra não tem duplo sentido:

        \bBlockQuote{% !j -i12 116
            Antes, como  \textbf{Deus  é  fiel},  a  \textbf{nossa  palavra}  para  convosco  \textbf{não  é  sim  e
            não}.}{2Co~1.18}{ARA}

    Ainda, corrobora com tais testemunhos:

        \bBlockQuote{% !j -i12 116
            Na verdade, \textbf{Deus não procede maliciosamente};}{Jó~34.12}{ARA}

    \noindent e também:

        \bBlockQuote{% !j -i12 116
            \textbf{Fez o Senhor o que intentou};}{Lm~2.17}{ARA}

    \noindent e ainda:

        \bBlockQuote{% !j -i12 116
            se somos infiéis, \textbf{ele permanece fiel},}{2Tm~2.13}{ARA}

    As Escrituras, portanto, abundantemente testificam da fidelidade, sinceridade, bondade e verdade de Deus,  cuja  Palavra  só
    pode significar exatamente o que diz. Deus não é falso, nem procede maliciosamente para intentar algo e falar diferentemente
    para expressar sua intenção. Assim:

    % !j 116 -i8
    \begin{PRI}[da Genuína Intenção de Deus]
        \label{pri.genuina}
        De acordo com as Escrituras, Deus não procede maliciosamente, antes, sempre genuinamente intenta  exatamente
        o que diz, como diz.
    \end{PRI}


    %---------------------------------------------------------------------------------------------------------------------------
    \subsection{Da Tradição de Deus}

    As Escrituras mostram Deus revelando-se a si mesmo e o seu plano, \emph{progressivamente}, ao longo da  história  humana.  A
    \emph{Torah}, ou, o \emph{Pentateuco} --- os cinco primeiros livros, de Moisés, de Gênesis a Deuteronômio --- já mostra isso
    claramente.

    Partindo de um: \bLineQuote{\textbf{maldita é a terra} por tua causa; em \textbf{fadigas} obterás dela o sustento durante os
    dias de tua vida}{Gn~3.17}{ARA}, observamos uma \emph{forte tradição oral entre osdescendentes de Adão}, pois  Lameque  diz,
    de seu filho Noé: \bLineQuote{Este nos consolará  dos  nossos  trabalhos  e  das  \textbf{fadigas}  de  nossas  mãos,  nesta
    \textbf{terra que o Senhor amaldiçoou}.}{Gn~5.28,29}{ARA}. Segundo as Escrituras em Gênesis 5, esta frase foi dita \emph{126
    anos após a morte de Adão}, ou \emph{1056 anos após a Criação}.

    Ainda, por causa de um \emph{prometido} \bLineQuote{descendente}{da mulher, de Gn~3.15}{ARA},  observamos  uma  tradição  de
    genealogias nas Escrituras em conexão com a humanidade, seguindo as revelações subsequentes feitas a  Abraão,  a  Isaque,  a
    Jacó, a Judá, a Davi, de Gênesis até \bLineQuote{Jesus Cristo, filho de Davi, filho de Abraão}{Mt~1.1}{ARA}.

    Neste processo de formação de uma ``tradição de Deus,'' certos \emph{elementos-chave,  estabelecidos  anteriormente},  viram
    \emph{referências} em falas e revelações futuras, como no caso das \bQuote{fadigas} e da ``maldição da terra,''  no  exemplo
    da fala de Lameque.

    Esta crescente ``tradição de Deus,'' com forte uso de \emph{referências anteriores} permeia as Escrituras e  é  determinante
    para uma interpretação ``segundo Deus,'' de passagens adiante. Este ponto  é  importante,  porque  podemos  ser  tentados  a
    empregar nossas definições, ao invés das de Deus, para termos-chave que aparecem depois, e assim errar,  estando  presumidos
    em nós mesmos, sem identificar a referência bíblica que está sendo feita ``segundo Deus.''

    Considere, por exemplo, a seguinte passagem:

        \bBlockQuote{% !j -i12 116
            atentando, diligentemente, por que ninguém seja faltoso, separando-se da graça de Deus; nem haja  alguma
            \textbf{raiz  de  amargura}  que,  brotando,  vos   perturbe,   e,   por   meio   dela,   muitos   sejam
            contaminados;}{Hb~12.15}{ARA}

    Seria esta uma exortação à não disseminação de \emph{sentimentos de amargura?}  Talvez  muitos,  presumidos  em  si  mesmos,
    concluam que sim, afinal o texto fala de \bQuote{raiz de amargura}!

    Porém, na tradição de Deus, o termo já possui definição, na Lei:

        \bBlockQuote{% !j -i12 116
            para que, entre vós, não haja homem, nem mulher, nem família, nem  tribo  cujo  \textbf{coração},  hoje,
            \textbf{se desvie do Senhor}, nosso Deus, e vá servir aos deuses destas nações; para que não haja  entre
            vós \textbf{raiz que produza erva venenosa e amarga},}{Dt~29.18}{ARA}

    Aplicando a tradição de Deus ao texto de Hb~12.15, torna sua exortação muito mais condizente, a saber: a não separar-se da
    graça de Deus nem seguir após outros falsos deuses, que pode contaminar a outros e perturbar o grupo.

    Ora, o princípio da tradição de Deus é bíblico, pois Deus, através de Paulo, diz:

        \bBlockQuote{% !j -i12 116
            Pois \textbf{tudo quanto, outrora, foi escrito para o nosso ensino foi escrito},  a  fim  de  que,  pela
            paciência e pela consolação das Escrituras, tenhamos esperança.}{Rm~15.4}{ARA}

    Aqui cabe acrescentar algo importante, para nós, que,  diferentemente  de  outras  épocas,  temos  \emph{acesso}  a  toda  a
    revelação, com os 66 livros da Bíblia: folhear a Bíblia é também `viajar no tempo' e, tendo toda a Bíblia  em
    mãos, devemos estar cientes da (i)~natureza progressiva da revelação, e que, (ii)~em cada época, as  referências  empregadas
    serviram  para  entendimento  dos  ouvintes,  \emph{à  época}!  Pois,  \bLineQuote{Na  verdade,  \textbf{Deus  não   procede
    maliciosamente}}{Jó~34.12}{ARA}; e assim, \emph{não falaria uma coisa, querendo dizer outra, com um  sentido  futuro,  ainda
    desconhecido da audiência a quem foi dirigida a Palavra}!
    
    Enuncia-se, então o seguinte princípio:

    % !j 116 -i8
    \begin{PRI}[da Dependência da Palavra]
        \label{pri.palavra}
        \bQuote{Tudo quanto, outrora, foi escrito para o nosso ensino foi escrito}. \bQuote{Confia no Senhor de todo
        o teu coração e não te estribes no  teu  próprio  entendimento};  \bQuote{adquire  a  sabedoria,  adquire  o
        entendimento e não te esqueças das palavras da minha boca, nem delas te apartes.}
    \end{PRI}

    O Princípio~\ref{pri.palavra} é uma compilação direta de Rm~15.4; Pv~3.5 e Pv~4.5; todos (ARA)~\cite{ARA}.


    %---------------------------------------------------------------------------------------------------------------------------
    \subsection{Do Direcionamento da Palavra de Deus}

    As Escrituras revelam que a Palavra de Deus é direcionada, primeiramente, à sua audiência original: \bLineQuote{Tendo  Jesus
    concluído todas as suas palavras dirigidas ao povo, entrou em Cafarnaum.}{Lc~7.1}{ARA}, e: \bLineQuote{Palavra do Senhor que
    foi  dirigida  a  Joel,  filho  de  Petuel.}{Jl~1.1}{ARA},  e  também:  \bLineQuote{Ao  falar  ele   \textbf{comigo}   estas
    palavras,}{Dn~10.15}{ARA}.

    Juntando-se a isso a natureza progressiva da revelação, bem como a Tradição de Deus, desenvolvidas na subseção anterior,
    pode-se extrair o princípio do direcionamento da Palavra:

    % !j 116 -i8
    \begin{PRI}[do Direcionamento da Palavra de Deus]
        \label{pri.direcionam}
        As Escrituras são primeiramente direcionadas à sua audiência original, de  acordo  com  a  revelação  a  ela
        disponível.
    \end{PRI}



% !center 124 | frame 124 -f'\%-\% '
%------------------------------------------------------------------------------------------------------------------------------%
%                                                     Algumas Implicações                                                      %
%------------------------------------------------------------------------------------------------------------------------------%

\section{Algumas Implicações}

    Buscar\footnote{Uma vez que os princípios identificados são indispensáveis (necessários), porém não  suficientes  a  estudos
    \bQuote{segundo Deus}, usa-se `buscar', uma vez que não se arroga aqui a `receita' de  tê-lo  alcançado.}  estudar  profecia
    ``segundo Deus,'' \emph{requer} reconhecer que Deus tem um plano \emph{único} e estabelecido,  segundo  o  conselho  de  Sua
    vontade, que este plano é \emph{verdadeiramente} revelado nas Escrituras, as quais, vindas de um  Deus  fiel  e  verdadeiro,
    constituem-se em \emph{promessas} nas quais devemos esperar, as quais, a seu devido tempo,  fielmente  \emph{cumprir-se-ão},
    tal que no futuro serão \emph{verificáveis}, que aconteceram  e  sucederam  \emph{como  profetizado}  por  um  Deus  fiel  e
    verdadeiro que \emph{intenta o que diz} em sua revelação progressiva, que \emph{nos ensina do início ao fim}.

    Agora que os príncípios bíblicos estão identificados e embasados nas Escrituras, desenha-se algumas implicações, uma vez que
    a violação dos princípios bíblicos constitui instrumento inequívoco e cabal para comprovação de escatologias com erro,  como
    discutido anteriormente, de imediato, reconhece-se as seguintes implicações:

    %!j -i12 -H-4 116
    \begin{enumerate}

        \item   Pelo   princípio   bíblico   da    \emph{Unicidade    da    Realidade    do    Início    ao    Fim},
            Princípio~\ref{pri.unicidade}, servos de Jesus Cristo não deveriam tolerar  a  existência  de  múltiplas
            `teorias proféticas' ou `linhas de interpretação escatológicas,' pois Deus, que anuncia \bQuote{o  fim},
            é o mesmo que exorta, através de Paulo, a que \bLineQuote{\emph{penseis a mesma coisa}}{Fp~2.2}{ARA}.

        \item Pelos princípios bíblicos da \emph{Veracidade das Profecias Divinas},  Princípio~\ref{pri.veracidade},
            da  \emph{Equivalência  Entre  Profecias  e   Promessas   Divinas},   Princípio~\ref{pri.promessa},   da
            \emph{Verificabilidade Das Profecias Divinas}, Princípio~\ref{pri.verificabilidade}, da \emph{Vigilância
            Divina  à  Sua  Palavra},  Princípio~\ref{pri.vigilância},  do  \emph{Cumprimento   Como   Profetizado},
            Princípio~\ref{pri.comoprof}, pelo Corolário~\ref{def.cor.alegorico}, da \emph{Alegação  de  Cumprimento
            Alegórico}, e pelo princípio bíblico da \emph{Genuína Intenção  de  Deus},  Princípio~\ref{pri.genuina},
            \textbf{quaisquer linhas de interpretação alegóricas de  profecias},  tal  que  passagens  bíblicas  não
            signifiquem o que nelas está escrito,  jamais  deveriam  sequer  ser  consideradas,  seja  acadêmica  ou
            devocionalmente, por servos do Senhor Jesus Cristo.  Pelo  contrário,  deveriam  ser  \textbf{cabalmente
            reprovadas e rejeitadas como pecado de rebelião contra o Senhor, nosso Deus e contra Sua Palavra!}

        \item Pelo princípio bíblico do \emph{Direcionamento da Palavra  de  Deus},  Princípio~\ref{pri.direcionam},
            linhas e argumentos de interpretação  que  recorrem  a  passagens  futuras  para  explicar  termos-chave
            empregados em passagens anteriores, tal que seu significado torne-se inacessível à audiência da passagem
            anterior, devem ser rejeitados, por servos  do  Senhor  Jesus  Cristo,  como  manipulação  indevida  das
            Escrituras.

    \end{enumerate}


% !center 124 | frame 124 -f'\%-\% '
%------------------------------------------------------------------------------------------------------------------------------%
%                                                         Conclusions                                                          %
%------------------------------------------------------------------------------------------------------------------------------%

\section{Conclusão}

    É visível na cristandade atual um estado indesejado de uma pluralidade de visões escatológicos, supostamente  bíblicas,  mas
    com irreconciliáveis incompatibilidades nas suas conclusões.

    Crendo que há uma forma \bLineQuote{segundo Deus, em justiça e retidão procedentes da verdade}{Ef~2.24}{ARA} de proceder com
    estudos escatológicos, que guiará seus estudantes, pelo Espírito de Deus, \bLineQuote{a toda a verdade;  porque  não  falará
    por si mesmo, mas dirá tudo o que tiver ouvido e vos anunciará as  coisas  que  hão  de  vir.}{Jo~16.13}{ARA},  este  estudo
    definiu   (a)~Escatologia   ``Segundo   Deus,''   Definição~\ref{def.esc.segDeus},   e   (b)~Escatologia    ``Com    Erro,''
    Definição~\ref{def.esc.comerro} --- cujas propriedades foram estabelecidas, bem como a importância, para membros do corpo de
    Cristo, de abandonar aquelas com erro e buscar e guardar as feitas \bQuote{segundo Deus}, as quais variam apenas em  escopo,
    sendo parte de uma único, são e coeso ensino \bQuote{segundo Deus}.

    Além das definições, foi proposto (i)~\emph{identificar} e (ii)~\emph{enunciar} princípios bíblicos indispensáveis a estudos
    \bQuote{segundo Deus} de profecias bíblicas, deduzidos à partir  das
    Escrituras, sob os axiomas de que as Escrituras são verdade e Palavra do único Deus verdadeiro. Foi arguído e demonstrado
    que tais princípios bíblicos podem ser empregados como \textsc{instrumentos para comprovação de escatologias ``com erro''},
    capazes de qualificar e expor, cabalmente o erro incorrido por uma dada escatologia sob análise.

    Apesar de não haver necessidade de completa identificação de tais princípios em um único estudo, este estudo  identificou  e
    enunciou 9 (nove) tais Princípios, um deles acompahado de um Corolário, a saber:

    \begin{itemize}\footnotesize

        \item Princípio~\ref{pri.unicidade}, da ``Unicidade da Realidade do Princípio ao Fim;''

        \item Princípio~\ref{pri.veracidade}, da ``Veracidade das Profecias Divinas;''

        \item Princípio~\ref{pri.promessa}, da ``Equivalência Entre Profecias e Promessas Divinas;''

        \item Princípio~\ref{pri.verificabilidade}, da ``Verificabilidade Das Profecias Divinas;''

        \item Princípio~\ref{pri.vigilância}, da ``Vigilância Divina à Sua Palavra;''

        \item Princípio~\ref{pri.comoprof}, do ``Cumprimento Como Profetizado;''

        \item Corolário~\ref{def.cor.alegorico}, da ``Alegação de Cumprimento Alegórico;''

        \item Princípio~\ref{pri.genuina}, da ``Genuína Intenção de Deus;''

        \item Princípio~\ref{pri.palavra}, da ``Dependência da Palavra;''

        \item Princípio~\ref{pri.direcionam}, do ``Direcionamento da Palavra de Deus.''

    \end{itemize}

    Implicações imediatas (1)~quanto ao atual e lamentável estado de  multiplicidade  de  visões  escatológicas;  (2)~quanto  às
    linhas de interpretação alegóricas de profecias, e (3)~àquelas que recorrem a passagens futuras para explicar termos-  chave
    empregados em passagens anteriores foram feitas, sem exaustão.

    Finalmente, crê-se também que \emph{qualquer} estudo Bíblico feito \bQuote{segundo Deus} levará à \emph{unidade da fé} a que
    somos exortados pelas Escrituras; pois que, conduzidas pelo Espírito que \bLineQuote{dá testemunho, porque o  Espírito  é  a
    verdade. E três são as testemunhas: o Espírito, a água e o sangue, e os três concordam.}{1Jo~5.7,8}{PESH}.


% !center 124 | frame 124 -f'\%-\% '
%------------------------------------------------------------------------------------------------------------------------------%
%                                                           Apêndice                                                           %
%------------------------------------------------------------------------------------------------------------------------------%

\appendix


% !center 124 | frame 124 -f'\%-\% '
%------------------------------------------------------------------------------------------------------------------------------%
%                                         Generalização Para Qualquer Doutrina Bíblica                                         %
%------------------------------------------------------------------------------------------------------------------------------%

\section{Generalização Para Qualquer Doutrina Bíblica}

    Como mencionado na Introdução, a estrutura conceitual das Definições, do  Teorema,  sua  Prova  e  Corolários,  apresentadas
    primeiramente no escopo da escatologia bíblica, é generalizável para \emph{qualquer} doutrina bíblica. Tal  generalização  é
    aqui apresentada:

    Tendo-se em mente abordagens bíblicas doutrinárias \bQuote{segundo Deus}, à semelhança e em generalização do que foi exposto
    e estabelecido dentro do escopo da escatologia, a saber, abordagens bíblicas doutrinárias feitas ``à maneira de  Deus;''
    aquelas que, baseadas unicamente  em verdade, são conduzidas em retidão e chegam à verdade, a saber, como de fato
    estabelecidas pelo próprio Deus:

    % !j 116 -i8
    \begin{DEF}[Doutrina ``Segundo Deus'']
        \label{def.dou.segDeus}
        A doutrina \bQuote{segundo Deus} é aquela feita \bLineQuote{segundo Deus, em justiça e  retidão  procedentes
        da verdade}{Ef~2.24}{ARA}.
    \end{DEF}

    Passa-se também à definição de doutrina com \emph{erro} --- entendido como qualquer violação direta de qualquer um  ou  mais
    preceitos das Escrituras --- ou doutrina com engano, ou enganosa, que não é segundo Deus:

    % !j 116 -i8
    \begin{DEF}[Doutrina Com Erro]
        \label{def.dou.comerro}
        Seja a doutrina com erro, aquela que incorre em erro, seja nas suas premissas, ou nos seus métodos,  ou  nos
        seus processos, ou nas suas conclusões.
    \end{DEF}

    % !j 116 -i8
    \begin{THE}[Dicotomia Doutrinária]
        \label{the.dou.dicotomia}
        Qualquer doutrina será ou \bQuote{segundo Deus}, ou ``com erro.''
    \end{THE}

    % !j 116 -i8
    \begin{proof}[Prova do Teorema da Dicotomia Doutrinária]
        \label{pro.the.dicotomia}

        Seja uma doutrina $\delta_i$ qualquer, com $i \in \{1,  2,  3,  \ldots\}$  sendo  um  índice  que  numera-a,
        identificando-a de forma única e inequívoca.

        Caso (i)~$\delta_i$ incorra  em  erro  (contenha  erro),  será,  pela  Definição~\ref{def.dou.comerro},  uma
        doutrina ``com erro.'' Neste caso, tal doutrina \emph{não  poderá}  ser  \bQuote{segundo  Deus},  pois  está
        escrito:

        \bBlockQuote{% !j -i12 116
            Não vos escrevi porque não saibais a verdade; antes, porque a sabeis, e  porque  \textbf{mentira  alguma
            jamais procede da verdade}.}{1Jo~2.21}{ARA}

        \noindent violando a Definição~\ref{def.dou.segDeus}, pelo erro jamais proceder da verdade.

        Caso (ii)~$\delta_i$ não contiver erro, não incorrer em erro, terá sido feita \bLineQuote{segundo  Deus,  em
        justiça    e    retidão    procedentes    da    verdade}{Ef~2.24}{ARA}    e     será,     portanto,     pela
        Definição~\ref{def.dou.segDeus}, doutrina \bQuote{segundo Deus}.

        Neste caso, tal doutrina não poderá  ser  ``com  erro,''  pois  a  premissa  de  não  conter  erro  viola  a
        Definição~\ref{def.dou.comerro}.

    \end{proof}

    % !j 116 -i8
    \begin{COR}
        \label{cor.dicotomia.1}
        Nenhuma doutrina pode ser simultaneamente \bQuote{segundo Deus} e ``com erro.''
    \end{COR}

    % !j 116 -i8
    \begin{COR}
        \label{cor.dicotomia.2}
        A união das Definições~\ref{def.dou.segDeus} e~\ref{def.dou.comerro} contém todas as doutrinas.
    \end{COR}

    As Definições~\ref{def.dou.segDeus} e~\ref{def.dou.comerro}  dadas  são  úteis  na  \emph{classificação}  de  doutrinas  que
    \textbf{devem ser abandonadas} por membros do corpo de Cristo (à luz de Ef~4.7,11--14, citado  na  Introdução),  e  aquelas,
    \bQuote{segundo Deus}, que \textbf{devem ser guardada} e também \emph{cridas, ensinadas e proclamadas}.


%------------------------------------------------------------------------------------------------------------------------------%
