% !center 124 | frame 124 -f'\%-\% '
%------------------------------------------------------------------------------------------------------------------------------%
%                                                          Title Page                                                          %
%------------------------------------------------------------------------------------------------------------------------------%

\thispagestyle{empty} % Removes page numbering from the first page
\flushbottom % Makes all text pages the same height
\maketitle % Print the title and abstract box

% !center 124 | frame 124 -f'\%-\% '
%------------------------------------------------------------------------------------------------------------------------------%
%                                                           Abstract                                                           %
%------------------------------------------------------------------------------------------------------------------------------%

\begin{abstract}
    Aqui vai o resumo.
\end{abstract}

% !center 124 | frame 124 -f'\%-\% '
%------------------------------------------------------------------------------------------------------------------------------%
%                                                           License                                                            %
%------------------------------------------------------------------------------------------------------------------------------%

\section*{Licença}

    \scriptsize\noindent%
    \begin{minipage}{\columnwidth}
        \centering\tt
        \includegraphics[height=6.0mm]{cc/by-nc.pdf}\\[0.5\smallskipamount]
        {\scriptsize\texttt{https://creativecommons.org/licenses/by-nc/4.0/}}
    \end{minipage}
    \normalsize


% !center 124 | frame 124 -f'\%-\% '
%------------------------------------------------------------------------------------------------------------------------------%
%                                                      Table Of Contents                                                       %
%------------------------------------------------------------------------------------------------------------------------------%

\tableofcontents


% !center 124 | frame 124 -f'\%-\% '
%------------------------------------------------------------------------------------------------------------------------------%
%                                                         Introduction                                                         %
%------------------------------------------------------------------------------------------------------------------------------%

\section{Introdução}

    Alguns recursos do modelo incluem citação de versos em bloco, via \verb!\bBlockQuote{Verso}{Loc~1.1}{ARA}!:

        \bBlockQuote{% !j -i12 116
            Ah! Que grande é aquele dia, e não há outro semelhante! É tempo de angústia para Jacó; ele, porém,  será
            livre dela.}{Jr~30.7}{ARA}

    Versos também podem ser citados em linha, ao longo de um parágrafo, via  \verb!\bLineQuote{Verso}{Loc~1.1}{ARA}!,  como  em:
    \bLineQuote{para sempre com o Senhor}{1Ts~4.17}{ARA}.

    Comandos de:

    \begin{itemize}
        \item \emph{ênfase},            \verb!\emph{ênfase}!,
        \item \textit{itálico},         \verb!\textit{itálico}!,
        \item \textbf{negrito},         \verb!\textbf{negrito}!,
        \item \texttt{typewritter},     \verb!\texttt{typewritter}!,
        \item \textsl{inclinado},       \verb!\textsl{inclinado}!,
        \item \textsw{Que (swoosh)},    \verb!\textsw{Que (swoosh)}! e
        \item \textsc{caixa alta},      \verb!\textsc{caixa alta}!,
    \end{itemize}

    funcionam melhor ou pior dependendo das fontes selecionadas como \verb!mainfont!, via \verb!\setmainfont!, no preâmbulo.

    O modelo é simples por projeto, para permitir foco no \emph{conteúdo}. Que Deus te abençoe!


% !center 124 | frame 124 -f'\%-\% '
%------------------------------------------------------------------------------------------------------------------------------%
%                                                         Conclusions                                                          %
%------------------------------------------------------------------------------------------------------------------------------%

\section{Conclusão}

    Conclusão, seguida de elementos pós-textuais.

%-------------------------------------------------------------------------------------------------------------------------------
